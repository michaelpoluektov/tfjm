%Ceci est le préambule du document. 
%J'aime bien le séparer pour garder le document principal lisible.
%On trouve ici:
%    les packages utilisés (pour permettre de faire des choses avancées)
%    des définitions personnelles pour se simplifier la vie.

 
%Type de document
\documentclass[11pt]{article}
%Encodage des caractères
\usepackage[utf8]{inputenc}

%Police
\usepackage{lmodern}
%Convention d'écriture française.
\usepackage[french]{babel}
\usepackage[T1]{fontenc}

%Pour les environnement théorèmes
\usepackage{amsthm}
%Le package de maths de base
\usepackage{amsmath}
% Un paquet de symboles mathématiques obscurs
\usepackage{amssymb}
%autres symboles
\usepackage{stmaryrd}
%Permet d'utiliser \mathscr{} pour les lettres calligraphiées
\usepackage{mathrsfs}



%pour insérer des images (png,jpg,etx)
\usepackage{graphicx}

%Permet d'utiliser l'environnement \begin{comment} ...\end{comment} pour faire de longs commentaires
%sans utiliser %
\usepackage{verbatim}


%Définition des marges du document
\usepackage[a4paper]{geometry}

% Couleurs dans les tableaux
\usepackage[table]{xcolor}

%Package de dessin avec ses sous-packages
\usepackage{tikz}
\usetikzlibrary{plotmarks}
%\usepackage{tikz-3dplot}
\usetikzlibrary{trees}
\usetikzlibrary{positioning}
\usetikzlibrary{decorations.pathreplacing}
\usetikzlibrary{shapes}
\usetikzlibrary{patterns}
\usetikzlibrary{through}
\usetikzlibrary{circuits}
\usetikzlibrary{calc}
\usetikzlibrary{arrows, automata}
\usetikzlibrary{babel} %compatibilité babel & pgf
\usetikzlibrary{arrows.meta}
\usepackage{tkz-tab} %tableau de variations
\usepackage{circuitikz}%circuits électriques
%\usepackage{pgfplots} %module de calcul avance dans les figures

%Pour afficher des sections dans la table des matières sans numéros$
\setcounter{secnumdepth}{0}

%Package de tableau meilleur que celui de base.
\usepackage{tabularx}

%permet de formatter du code
\usepackage{listings}
%sp
\lstnewenvironment{python}{\lstset{
     literate=%
         {à}{{\`a}}1
         {è}{{\`e}}1
         {é}{{\'e}}1
         {ç}{{\c c}}1,language=Python,frame=single,numbers=left,numbersep=5pt,tabsize=4}
         }{}

 % affichage correct des nombres avec séparateurs décimaux, comme \np{10000}
\usepackage[np]{numprint}

%option d'alignement m et b dans les tableaux
\usepackage{array}

%Pour mettre des légendes aux figures sans utiliser l'environnement figure
\usepackage{caption}

%Pour moi, pour mettre du verbatim dans les footnotes
%\usepackage{fancyvrb}
%\VerbatimFootnotes

%pour faire de jolies boites. Nécessite boiboites.sty
\usepackage{boiboites}
%Mon environnement théorème
\newboxedtheorem[boxcolor=red, background=blue!5, titlebackground=blue!20,titleboxcolor = black]{theorem}{Théorème}{anything}
%Je rajoute un énoncé pour le travail de l'été  (JT)
\newboxedtheorem[boxcolor=red, background=blue!5, titlebackground=blue!20,titleboxcolor = black]{enonce}{Enoncé}{anything}
% Et pour les figures
\newboxedtheorem[boxcolor=red,  background=blue!1, titlebackground=blue!20,titleboxcolor = black]{figureleg}{Figure}{anything}
%Mon environnement de preuve :
\usepackage{mdframed}%Pour l'environnement preuve
%définir la barre verticale:
\newmdenv[
  topline=false,
  bottomline=false,
	rightline=false,
  skipabove=\topsep,
  skipbelow=0pt,
	linecolor=red,
	leftmargin=0pt,
	innerleftmargin=5pt,
	innerbottommargin=0pt
]{leftrule}
\newcommand{\cqfd}{\hfill $\square$ \medskip}
\newenvironment{preuve}{
\noindent\emph{Démonstration.}
\begin{leftrule}
}{

\vspace{-.3cm}
\cqfd
\end{leftrule}
}

 % pour gérer les crédits et les liens hypertextes
%\usepackage[pdftex,colorlinks=true,linkcolor=blue,citecolor=blue,urlcolor=blue]{hyperref}
%\hypersetup{allbordercolors=white,pdfauthor={Lycée international de Ferney-Voltaire}}

% Pour des jolis hauts de pages
\usepackage{fancyhdr}
\pagestyle{fancy}
\lhead{\today}
\chead{}
\rhead{Equipe $\Pi$rates du Gex Axis (PGA)}
\lfoot{}
\cfoot{\thepage}
\rfoot{}

% pour les marges
\textwidth=18cm %15.1cm
\headwidth = \textwidth
 \textheight=25.5cm %21.8cm
 \topmargin=-1.5cm 
 \headheight=0.5cm
 \headsep=0.7cm %1.2cm
 \oddsidemargin=-1cm
 \evensidemargin=-1cm
 \marginparwidth=0cm



%Pour avoir des environnements definition, remarque et exemple. Les * retirent la numérotation.
\newtheorem{definition}{Définition}
\newtheorem*{remarque}{Remarque}
\newtheorem*{exemple}{Exemple}

%Pour que les sections soient numérotées de la forme I.1).a
\renewcommand{\thesection}{\Roman{section}.}
\renewcommand{\thesubsection}{\arabic{subsection})}
\renewcommand{\thesubsubsection}{\alph{subsubsection})}

%%%%%%%%%%%%%%%%%%%%%%%%%%%%
%%%%%%%%%%%%%%%%%%%%%%%%%%%%
%%%%%%%Bourbaki%%%%%%%%%%%%%
%%%%%%%%%%%%%%%%%%%%%%%%%%%%
%%%%%%%%%%%%%%%%%%%%%%%%%%%%

%From bourbaki.sty, 1995/05/20 v0.1 : Mode Bourbaki
%pour avoir les lettres droites en mode maths.

\DeclareMathSymbol{A}{\mathalpha}{operators}{`A}%
\DeclareMathSymbol{B}{\mathalpha}{operators}{`B}%
\DeclareMathSymbol{C}{\mathalpha}{operators}{`C}%
\DeclareMathSymbol{D}{\mathalpha}{operators}{`D}%
\DeclareMathSymbol{E}{\mathalpha}{operators}{`E}%
\DeclareMathSymbol{F}{\mathalpha}{operators}{`F}%
\DeclareMathSymbol{G}{\mathalpha}{operators}{`G}%
\DeclareMathSymbol{H}{\mathalpha}{operators}{`H}%
\DeclareMathSymbol{I}{\mathalpha}{operators}{`I}%
\DeclareMathSymbol{J}{\mathalpha}{operators}{`J}%
\DeclareMathSymbol{K}{\mathalpha}{operators}{`K}%
\DeclareMathSymbol{L}{\mathalpha}{operators}{`L}%
\DeclareMathSymbol{M}{\mathalpha}{operators}{`M}%
\DeclareMathSymbol{N}{\mathalpha}{operators}{`N}%
\DeclareMathSymbol{O}{\mathalpha}{operators}{`O}%
\DeclareMathSymbol{P}{\mathalpha}{operators}{`P}%
\DeclareMathSymbol{Q}{\mathalpha}{operators}{`Q}%
\DeclareMathSymbol{R}{\mathalpha}{operators}{`R}%
\DeclareMathSymbol{S}{\mathalpha}{operators}{`S}%
\DeclareMathSymbol{T}{\mathalpha}{operators}{`T}%
\DeclareMathSymbol{U}{\mathalpha}{operators}{`U}%
\DeclareMathSymbol{V}{\mathalpha}{operators}{`V}%
\DeclareMathSymbol{W}{\mathalpha}{operators}{`W}%
\DeclareMathSymbol{X}{\mathalpha}{operators}{`X}%
\DeclareMathSymbol{Y}{\mathalpha}{operators}{`Y}%
\DeclareMathSymbol{Z}{\mathalpha}{operators}{`Z}%

%Quelques notations
\def\N{{\mathbb N}}
\def\Z{{\mathbb Z}}
\def\R{{\mathbb R}}
\def\C{{\mathbb C}}
\def\Q{{\mathbb Q}}
\def\P{{\mathscr P}}
\newcommand{\pI}{+\infty}
\newcommand{\mI}{-\infty}
\newcommand{\abs}[1]{\left| #1 \right|}
\def\eps{\varepsilon}
%Vecteurs
\newcommand{\vvec}[1]{
{\overrightarrow{#1}}
}

%Pour facilier l'écriture des limites
\newcommand{\limseq}{\lim\limits_{n\to \pI}}
\newcommand{\lims}{\lim\limits}

%Pour faciliter l'écriture des systèmes: 
\newenvironment{systeme}{\left\{\begin{array}{l}}{\end{array}\right.}

%Parenthèses avec left-right intégrés
\renewcommand{\(}{\left(}
\renewcommand{\)}{\right)}

%Pour que les tirets soient remplacés par des textbullet
\AtBeginDocument{
  \renewcommand\labelitemi{\textbullet}
}

%Pour définir des couleurs personnalisées
\usepackage{color}
\definecolor{orangekamil}{rgb}{1,0.3,0}  % r:red  g:green   b:blue avec des valeurs entre 0 et 1

%Pour les commentaires prof
\newcommand{\comjt}[1]{\textcolor{pink}{#1}}
\newcommand{\comjb}[1]{\textcolor{green}{#1}}

%Pour les commentaires élèves 
\newcommand{\comg}[1]{\textcolor{orangekamil}{#1}} % Gessienne
\newcommand{\comj}[1]{\textcolor{yellow}{#1}} % Jason
\newcommand{\comm}[1]{\textcolor{red}{#1}} % Michael
\newcommand{\comn}[1]{\textcolor{purple}{#1}} % Natasha 
\newcommand{\comro}[1]{\textcolor{blue}{#1}} % Roman
\newcommand{\comru}[1]{\textcolor{brown}{#1}} % Ruben


